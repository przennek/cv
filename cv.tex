% Based on: Matty's resume from overleaf
% which is based on https://github.com/jakegut/resume
% which was itself based on https://github.com/sb2nov/resume

\documentclass[letterpaper,11pt]{article}

\usepackage{latexsym}
\usepackage[empty]{fullpage}
\usepackage{titlesec}
\usepackage{marvosym}
\usepackage[usenames,dvipsnames]{color}
\usepackage{verbatim}
\usepackage{enumitem}
\usepackage[hidelinks]{hyperref}
\usepackage{fancyhdr}
\usepackage[english]{babel}
\usepackage{tabularx}
\usepackage{xcolor}
\usepackage{fontawesome5}
\usepackage{geometry}
\usepackage{ragged2e}

\input{glyphtounicode}

% -------------------- FONT OPTIONS --------------------
% sans-serif
% \usepackage[sfdefault]{roboto}
% \usepackage[sfdefault]{noto-sans}
% serif
% \usepackage{charter}

\pagestyle{fancy}
\fancyhf{} % clear all header and footer fields
\fancyfoot{}
\renewcommand{\headrulewidth}{0pt}
\renewcommand{\footrulewidth}{0pt}

% Adjust margins
\addtolength{\oddsidemargin}{-0.5in}
\addtolength{\evensidemargin}{-0.5in}
\addtolength{\textwidth}{1in}
\addtolength{\topmargin}{-1in} % Default was -.5in
\addtolength{\textheight}{1.0in}

\urlstyle{same}

\raggedbottom
\raggedright
\setlength{\tabcolsep}{0in}

% Section formatting
\titleformat{\section}{
  \vspace{-5pt}\scshape\raggedright\large
}{}{0em}{}[\color{black}\titlerule \vspace{-5pt}]

% Subsection formatting
\titleformat{\subsection}{
  \vspace{-4pt}\scshape\raggedright\large
}{\hspace{-.15in}}{0em}{}[\color{black}\vspace{-8pt}]

% Ensure that generate pdf is machine readable/ATS parsable
\pdfgentounicode=1

% -------------------- CUSTOM COMMANDS --------------------
\newcommand{\resumeItem}[1]{
  \item\small{
    {#1 \vspace{-2pt}}
  }
}

\newcommand{\resumeSubheading}[4]{
  \vspace{-2pt}\item
    \begin{tabular*}{0.97\textwidth}[t]{l@{\extracolsep{\fill}}r}
      \textbf{#1} & #2 \\
      \textit{\small#3} & \textit{\small #4} \\
    \end{tabular*}\vspace{-7pt}
}

\newcommand{\resumeSubSubheading}[2]{
    \item
    \begin{tabular*}{0.97\textwidth}{l@{\extracolsep{\fill}}r}
      \textit{\small#1} & \textit{\small #2} \\
    \end{tabular*}\vspace{-7pt}
}

\newcommand{\resumeProjectHeading}[2]{
    \item
    \begin{tabular*}{0.97\textwidth}{l@{\extracolsep{\fill}}r}
      \small#1 & #2 \\
    \end{tabular*}\vspace{-7pt}
}

\newcommand{\resumeSubItem}[1]{\resumeItem{#1}\vspace{-4pt}}
\newcommand{\resumeSubHeadingListStart}{\begin{itemize}[leftmargin=0.15in, label={}]}
\newcommand{\resumeSubHeadingListEnd}{\end{itemize}}
\newcommand{\resumeItemListStart}{\begin{itemize}}
\newcommand{\resumeItemListEnd}{\end{itemize}\vspace{-5pt}}

\renewcommand\labelitemii{$\vcenter{\hbox{\tiny$\bullet$}}$}

\setlength{\footskip}{4.08003pt}

% margins
\geometry{
    letterpaper,
    top=1in,       % Adjust as needed
    bottom=1.2in,    % Adjust as needed
    left=0.75in,   % Adjust as needed
    right=0.75in   % Adjust as needed
}

% -------------------- START OF DOCUMENT --------------------
\begin{document}

% -------------------- HEADING--------------------

\begin{center}
    \textbf{\Huge \scshape Przemysław Chlipała} \\ \vspace{8pt}
    \small 
    \faIcon{github}
    \href{https://github.com/przennek}{\underline{github.com/przennek}} $  $
    \faIcon{linkedin}
    \href{https://www.linkedin.com/in/przemys%C5%82aw-chlipa%C5%82a-712aa7106/}{\underline{linkedin}} $  $
    \faIcon{envelope}
    \href{mailto:przemyslaw.chlipala@gmail.com}
    {\underline{przemyslaw.chlipala@gmail.com}} $ $
    \faIcon{graduation-cap}
    \href{https://www.coursera.org/user/2f79a023ccc47ab66ca3c0ca67fd77aa}{\underline{coursera}}
\end{center}
% -------------------- EDUCATION --------------------
\section{Education}
  \resumeSubHeadingListStart
  
    \resumeSubheading
      {Jagiellonian University}{2017 - 2019}
      {M.S. Computer Science}{\href{https://github.com/przennek/neurons_activation_visualisation/blob/master/latex/praca.pdf}{thesis: 'Visualisation methods of artificial neurons' activation optimization.'}}

    \resumeSubheading
      {AGH University}{2013 - 2017}
      {B.S. Applied Computer Science}{\href{https://misio.fis.agh.edu.pl/media/misiofiles/f697b33ada0bf415d95135a7010466a9.pdf}{thesis: 'A microservice application for managing tests.'}}
      
  \resumeSubHeadingListEnd
% -------------------- EXPERIENCE --------------------
\section{Experience}
  \resumeSubHeadingListStart
        \resumeProjectHeading
          {\textbf{Evoke} $|$ \footnotesize\emph{Senior Software Developer (fullstack)}\vspace{8pt}}{July 2024 -- now}
          {\small{
            \begin{itemize}                
                \item \justifying Know Your Customer (KYC) business logic development. Refactored workflows using Project Reactor, enabling high-concurrency support and improved scalability. Migrated components from a blocking to a non-blocking architecture, enhancing resource utilization and enabling the system to handle high-concurrency workloads. Designed solutions across various system areas based on complex business requirements. Provided technical mentorship to team members. Leveraged New Relic dashboards for application monitoring, debugging, and performance insights. \\
                \textit{Key technologies: Java, Spring, GraphQL, Docker, Terraform, AWS, Splunk, New Relic.}

            \end{itemize}
 
            }
          }
          \resumeProjectHeading
          {\textbf{Akamai Technologies} $|$ \footnotesize\emph{Senior Software Engineer}\vspace{8pt}}{December 2020 -- June 2023}
          {\small{
            \begin{itemize}                
                \item \justifying Built an IPSEC tunnels configuration service that generated and applied NETCONF-based configurations on the SRX. These tunnels were part of a larger solution that enabled mobile network users to access protected services without client-side VPN. Additionally, developed an SNMP-based monitoring service to track the status of IPSEC connections. Responsibilities included monitoring, testing, and implementing automated deployments. \\
                \textit{Key technologies: Java, Spring (MVC, data JPA, test, etc.), Docker, Kubernetes, Terraform, Microsoft Azure (queues, CosmosDB, Devops), JUnit5, Graphana, logzio.}

                \item Developed an authorization and authentication service called SPSON, used by mobile phones to access the secure DNS Server over DoH. Application covered REST API key-exchange services between on-premises DNS servers and a cloud-based SPSON, including APIs for token distribution and revocation to mobile devices. Implemented HMAC-based authorization for the authentication service, automated key rotation for over a million devices, conducted load testing, performance tuning, and implemented monitoring with alerting. Additionally, developed a frontend for operations (OPS). \\
                \textit{Key technologies: Python, Flask, Docker, Cassandra, Bouncy Castle, locust, Graphana, vuejs, AWS.}
            \end{itemize}
 
            }
          }
          
          \resumeProjectHeading
          {\textbf{Akamai Technologies} $|$ \footnotesize\emph{Software Engineer II}\vspace{8pt}}{May 2019 -- November 2020}
          {\small{
            \begin{itemize}
                \item \justifying
Development within a website access restriction application used for parental control. The application portal is responsible for loading and monitoring the status of modules used by the application. \\
                 \textit{Key technologies: Java, Spring, Hibernate, Docker, Kafka, PostgreSQL, AngularJS.}
                \item \justifying Developed a reactive web application for a campaign information pop-up utility used by the ISPs. \\ 
                \textit{Key technologies: Java, Spring (test, data JDBC), Docker, PostgreSQL.}
            \end{itemize}
            }
            }

          \resumeProjectHeading
          {\textbf{Akamai Technologies} $|$ \footnotesize\emph{Software Engineer}\vspace{8pt}}{June 2017 -- April 2019}
          {\small{
            \begin{itemize}
                \item \justifying Developed a managed content delivery network (MCDN) notification app that aggregated data from multiple sources and implemented logic to determine whether there were changes in the network configuration. \\
                \textit{Key technologies: Spring, Mybatis, PostgreSQL, Docker }
            \end{itemize}
            }

          \resumeProjectHeading
          {\textbf{Sabre} $|$ \footnotesize\emph{Java Software Developer}\vspace{8pt}}{March 2017 -- May 2017}
          {\small{
          \begin{itemize}
              \item \justifying Rewriting an AutoSys jobs based application to a Spring based REST API. The application aimed to rebalance the contents of airline warehouses and provide detailed information on necessary stock adjustments.
          \end{itemize} 
}}

          \resumeProjectHeading
          {\textbf{Sabre} $|$ \footnotesize\emph{Software Engineer in tests associate}\vspace{8pt}}{October 2015 -- February 2017}
          {\small{
          \begin{itemize}
              \item \justifying Development of a test framework for automated end-to-end backend testing. Developed automated tests for the IX platform, including regression test automation and execution, as well as preparation of test case scenarios.
          \end{itemize}

}}

          \resumeProjectHeading
          {\textbf{Sabre} $|$ \footnotesize\emph{Intern}\vspace{8pt}}{July 2015 -- September 2015}
          {\small{
          \begin{itemize}
              \item \justifying Development of an E2E automated tests for the
Inteligence Exchange (IX) platform. Development of an application for a health check monitoring of said platform.
          \end{itemize}
}}

\resumeSubHeadingListEnd

% -------------------- SKILLS --------------------
\section{Skills}
\begin{itemize}[leftmargin=0.15in, label={}]
\small{\item{
    \textbf{Languages:} Java (primary), Python, SQL \\
    \textbf{Frameworks:} Spring (Boot, MVC, Data, Security, Test) \\
    \textbf{DevOps:} Docker, Kubernetes \\
    \textbf{Cloud:} AWS, cloud monitoring tools \\
    \textbf{Data Stores:} PostgreSQL, Cassandra, CosmosDB \\
    \textbf{CI/CD:} Jenkins, Ansible, Git \\
    \textbf{Testing:} unit and integration testing, load testing, end-to-end automation \\
    \textbf{Environments:} Linux, IntelliJ IDEA, shell scripting \\
}}
\end{itemize}


% -------------------- Lang --------------------
\section{Spoken languages}
 \begin{itemize}[leftmargin=0.15in, label={}]
    \small{\item{
     \textbf{Polish}{: native} \\
     \textbf{English}{: full professional proficiency} \\
    }}
 \end{itemize}

 % -------------------- Train --------------------
\section{Trainings}
  \resumeSubHeadingListStart
    \resumeSubSubheading
        {Kubernetes}{(Akamai/Chmurowisko) 2023}
    \resumeSubSubheading
        {Deep Learning}{(Coursera/deeplearning.ai) 2019}
    \resumeSubSubheading
        {Presentation skills advanced}{(Akamai) 2019}
    \resumeSubSubheading
        {Modern Business e-Correspondence}{(Akamai) 2018}
    \resumeSubSubheading
        {Stress management}{(Akamai) 2018}
    \resumeSubHeadingListEnd


% RODO
\cfoot{\fontsize{9}{10}\selectfont \color{lightgray} I hereby give consent for my personal data included in my application to be processed for the purposes of the recruitment process.}

\end{document}
